\documentclass[mingoth,11pt,a4j,uplatex]{jsarticle}
\usepackage[top=20truemm,bottom=20truemm,left=20truemm,right=20truemm]{geometry}
\usepackage{moreverb}
\usepackage{listings,jlisting} %日本語のコメントアウトをする場合jlistingが必要
								% https://qiita.com/N_Matsukiyo/items/1199f07a0e1bf4fce29c

% https://qiita.com/ta_b0_/items/2619d5927492edbb5b03
\lstset{
  basicstyle={\ttfamily},
  identifierstyle={\small},
  commentstyle={\smallitshape},
  keywordstyle={\small\bfseries},
  ndkeywordstyle={\small},
  stringstyle={\small\ttfamily},
  frame={tb},
  breaklines=true,
  columns=[l]{fullflexible},
  numbers=left,
  xrightmargin=0zw,
  xleftmargin=3zw,
  numberstyle={\scriptsize},
  stepnumber=1,
  numbersep=1zw,
  lineskip=-0.5ex
}


\renewenvironment{description}%  descriptionをインデント
{%
   \begin{list}{\parbox{1zw}{$\bullet$}}% 見出し記号/直後の空白を調節
   {%
      \setlength{\topsep}{1zh}
      \setlength{\itemindent}{3zw}
      \setlength{\leftmargin}{5zw}%  左のインデント
      \setlength{\rightmargin}{0zw}% 右のインデント
      \setlength{\labelsep}{1zw}%    黒丸と説明文の間
      \setlength{\labelwidth}{3zw}%  ラベルの幅
      \setlength{\itemsep}{0em}%     項目ごとの改行幅
      \setlength{\parsep}{0em}%      段落での改行幅
      \setlength{\listparindent}{0zw}% 段落での一字下り
   }
}{%
   \end{list}%
}

\title{Canvas First Step}
%\author{小林 統 \thanks{帝京平成大学現代ライフ学部人間文化学科メディア文化コース}}
\date{\today}

\setcounter{secnumdepth}{3}
\setcounter{tocdepth}{3}

\begin{document}
%\gtfamily	%全てゴシックに

\maketitle

\begin{abstract}
WebGLを扱いやすくしたthree.jsを利用して3Dアニメーションを作成してみよう。
\end{abstract}

\tableofcontents
\newpage

\section{はじめに}
\subsection{読み間違えないでね}

\begin{lstlisting}[caption=読み間違えないでね]
数字:0123456789
小文字:abcdefghijklmnopqrstuvwxyz
大文字:ABCDEFGHIJKLMNOPQRSTUVWXYZ

1:イチ
l:小文字のエル
i:小文字のアイ
!:ビックリマーク
|:バーティカルバー。Shiftと¥を押したもの。

0:ゼロ
o:小文字のオー
O:大文字のオー

.:ピリオド
,:コンマ
\end{lstlisting}

\subsection{注意}
\begin{itemize}
\item これから出てくるソースコードには、左に「行番号」と呼ばれる番号が出てくるけど、入力する必要ないからね。

\item scriptタグの中で「//」で始まる文は、コメントで、プログラムは読み飛ばすよ。

\item コピーできるところはコピーして効率よく入力して行こう
\item 徐々に追加されていくから、量が多く見えるけど、平気だよ!
\end{itemize}


\newpage
\section{three.js入門}
参考:https://ics.media/tutorial-three/

\subsection{05-11.html 立方体を配置してみよう}
\begin{lstlisting}[caption=05-11.html]
<!DOCTYPE html>
<html>
  <head>
  <meta charset="utf-8" />
  <script src="https://unpkg.com/three@0.137.4/build/three.min.js"></script>
  <script>
    // ページの読み込みを待つ
    window.addEventListener('DOMContentLoaded', init);

    function init() {
      // サイズを指定
      const width = 960;
      const height = 540;

      // レンダラーを作成
      const renderer = new THREE.WebGLRenderer({
        canvas: document.querySelector('#myCanvas'),
      });
      renderer.setPixelRatio(window.devicePixelRatio);
      renderer.setSize(width, height);

      // シーンを作成
      const scene = new THREE.Scene();

      // カメラを作成
      const camera = new THREE.PerspectiveCamera(45, width / height);
      camera.position.set(0, 0, +1000);

      // 箱を作成
      const geometry = new THREE.BoxGeometry(400, 400, 400);
      const material = new THREE.MeshNormalMaterial();
      const box = new THREE.Mesh(geometry, material);
      scene.add(box);

      // レンダリング
      renderer.render(scene, camera); 
    }
  </script>
</head>
<body>
    <canvas id="myCanvas"></canvas>
  </body>
</html>
\end{lstlisting}

正方形が見えるだけで、立方体だってわからないね。

\subsection{05-12.html 立方体を回してみよう}
\begin{lstlisting}[caption=05-12.html]
<!DOCTYPE html>
<html>
<head>
  <meta charset="utf-8"/>
  <script src="https://unpkg.com/three@0.137.4/build/three.min.js"></script>
  <script>
    // ページの読み込みを待つ
    window.addEventListener('DOMContentLoaded', init);

    function init() {

      // サイズを指定
      const width = 960;
      const height = 540;

      // レンダラーを作成
      const renderer = new THREE.WebGLRenderer({
        canvas: document.querySelector('#myCanvas')
      });
      renderer.setPixelRatio(window.devicePixelRatio);
      renderer.setSize(width, height);

      // シーンを作成
      const scene = new THREE.Scene();

      // カメラを作成
      const camera = new THREE.PerspectiveCamera(45, width / height);
      camera.position.set(0, 0, +1000);

      // 箱を作成
      const geometry = new THREE.BoxGeometry(400, 400, 400);
      const material = new THREE.MeshNormalMaterial();
      const box = new THREE.Mesh(geometry, material);
      scene.add(box);

      tick();

      // 毎フレーム時に実行されるループイベントです
      function tick() {
        box.rotation.y += 0.01;
        renderer.render(scene, camera); // レンダリング

        requestAnimationFrame(tick);
      }
    }
  </script>
</head>
<body>
  <canvas id="myCanvas"></canvas>
</body>
</html>
\end{lstlisting}

\subsection{05-13.html インタラクティブ性を追加しよう}
6行目に注意
\begin{lstlisting}[caption=05-13.html]
<!DOCTYPE html>
<html>
<head>
  <meta charset="utf-8"/>
  <script src="https://unpkg.com/three@0.137.4/build/three.min.js"></script>
  <script src="https://unpkg.com/three@0.137.4/examples/js/controls/OrbitControls.js"></script>
  <script>
    // ページの読み込みを待つ
    window.addEventListener('DOMContentLoaded', init);

    function init() {

      // サイズを指定
      const width = 960;
      const height = 540;

      // レンダラーを作成
      const renderer = new THREE.WebGLRenderer({
        canvas: document.querySelector('#myCanvas')
      });
      renderer.setPixelRatio(window.devicePixelRatio);
      renderer.setSize(width, height);

      // シーンを作成
      const scene = new THREE.Scene();

      // カメラを作成
      const camera = new THREE.PerspectiveCamera(45, width / height);
      camera.position.set(0, 0, +1000);

      // カメラコントローラーを作成
      const controls = new THREE.OrbitControls(camera, document.body);

      // 箱を作成
      const geometry = new THREE.BoxGeometry(400, 400, 400);
      const material = new THREE.MeshNormalMaterial();
      const box = new THREE.Mesh(geometry, material);
      scene.add(box);

      tick();

      // 毎フレーム時に実行されるループイベントです
      function tick() {
        box.rotation.y += 0.01;
        renderer.render(scene, camera); // レンダリング

        requestAnimationFrame(tick);
      }
    }
  </script>
</head>
<body>
  <canvas id="myCanvas"></canvas>
</body>
</html>
\end{lstlisting}

\subsection{05--21.html ライトを当ててみよう}
05-12.htmlを改造したほうがいいかも
\begin{lstlisting}[caption=05-21.html]
<!DOCTYPE html>
<html>
  <head>
    <meta charset="utf-8" />
    <script src="https://unpkg.com/three@0.137.4/build/three.min.js"></script>
    <script>
      // ページの読み込みを待つ
      window.addEventListener('DOMContentLoaded', init);

      function init() {
        // サイズを指定
        const width = 960;
        const height = 540;

        // レンダラーを作成
        const renderer = new THREE.WebGLRenderer({
          canvas: document.querySelector('#myCanvas'),
        });
        renderer.setSize(width, height);

        // シーンを作成
        const scene = new THREE.Scene();

        // カメラを作成
        const camera = new THREE.PerspectiveCamera(45, width / height);
        camera.position.set(0, 0, +1000);

        // 球体を作成
        const geometry = new THREE.SphereGeometry(300, 30, 30);
        // マテリアルを作成
        const material = new THREE.MeshStandardMaterial({ color: 0xff0000 });
        // メッシュを作成
        const mesh = new THREE.Mesh(geometry, material);
        // 3D空間にメッシュを追加
        scene.add(mesh);

        // 平行光源
        const directionalLight = new THREE.DirectionalLight(0xffffff);
        directionalLight.position.set(1, 1, 1);
        // シーンに追加
        scene.add(directionalLight);

        tick();

        // 毎フレーム時に実行されるループイベントです
        function tick() {
          // レンダリング
          renderer.render(scene, camera);

          requestAnimationFrame(tick);
        }
      }
    </script>
  </head>
  <body>
    <canvas id="myCanvas"></canvas>
  </body>
</html>
\end{lstlisting}

\subsection{05-22.html 球に画像を貼ってみよう}
http://planetpixelemporium.com/earth.html
から、color mapをダウンロードして、img/earthmap1k.jpgに配置しておこう

\begin{lstlisting}[caption=一定時間ごとにランダムな長方形を描こう]
<!DOCTYPE html>
<html>
<head>
    <meta charset="utf-8"/>
    <script src="https://unpkg.com/three@0.137.4/build/three.min.js"></script>
    <script>
        // ページの読み込みを待つ
        window.addEventListener('DOMContentLoaded', init);

        // サイズを指定
        const width = 960;
        const height = 540;

        function init() {
            // レンダラーを作成
            const renderer = new THREE.WebGLRenderer({
                canvas: document.querySelector('#myCanvas')
            });
            renderer.setSize(width, height);

            // シーンを作成
            const scene = new THREE.Scene();

            // カメラを作成
            const camera = new THREE.PerspectiveCamera(45, width / height, 1, 10000);
            camera.position.set(0, 0, +1000);

            // 球体を作成
            const geometry = new THREE.SphereGeometry(300, 30, 30);
            // 画像を読み込む
            const loader = new THREE.TextureLoader();
            const texture = loader.load('img/earthmap1k.jpg');
            // マテリアルにテクスチャーを設定
            const material = new THREE.MeshStandardMaterial({
                map: texture
            });
            // メッシュを作成
            const mesh = new THREE.Mesh(geometry, material);
            // 3D空間にメッシュを追加
            scene.add(mesh);

            // 平行光源
            const directionalLight = new THREE.DirectionalLight(0xFFFFFF);
            directionalLight.position.set(1, 1, 1);
            // シーンに追加
            scene.add(directionalLight);

            tick();

            // 毎フレーム時に実行されるループイベントです
            function tick() {
                // メッシュを回転させる
                mesh.rotation.y += 0.01;
                // レンダリング
                renderer.render(scene, camera);

            requestAnimationFrame(tick);
            }
        }
    </script>
</head>
<body>
    <canvas id="myCanvas"></canvas>
</body>
</html>
\end{lstlisting}

\subsection{05-23.html マウスのX軸によって見える方向を変えよう}


\begin{lstlisting}[caption=05-23.html]
<!DOCTYPE html>
<html>
<head>
    <meta charset="utf-8"/>
    <script src="https://unpkg.com/three@0.137.4/build/three.min.js"></script>
    <script>
        // ページの読み込みを待つ
        window.addEventListener('DOMContentLoaded', init);

        // サイズを指定
        const width = 960;
        const height = 540;

        function init() {
            let rot = 0; // 角度
            let mouseX = 0; // マウス座標

            // マウス座標はマウスが動いた時のみ取得できる
                document.addEventListener("mousemove", (event) => {
                mouseX = event.pageX;
            });

            // レンダラーを作成
            const renderer = new THREE.WebGLRenderer({
                canvas: document.querySelector('#myCanvas')
            });
            renderer.setSize(width, height);

            // シーンを作成
            const scene = new THREE.Scene();

            // カメラを作成
            const camera = new THREE.PerspectiveCamera(45, width / height, 1, 10000);
            camera.position.set(0, 0, +1000);

            // 球体を作成
            const geometry = new THREE.SphereGeometry(300, 30, 30);
            // 画像を読み込む
            const loader = new THREE.TextureLoader();
            const texture = loader.load('img/earthmap1k.jpg');
            // マテリアルにテクスチャーを設定
            const material = new THREE.MeshStandardMaterial({
                map: texture
            });
            // メッシュを作成
            const mesh = new THREE.Mesh(geometry, material);
            // 3D空間にメッシュを追加
            scene.add(mesh);

            // 平行光源
            const directionalLight = new THREE.DirectionalLight(0xFFFFFF);
            directionalLight.position.set(1, 1, 1);
            // シーンに追加
            scene.add(directionalLight);

            tick();

            // 毎フレーム時に実行されるループイベントです
            function tick() {
                // メッシュを回転させる
                mesh.rotation.y += 0.01;

                // マウスの位置に応じて角度を設定
                // マウスのX座標がステージの幅の何%の位置にあるか調べてそれを360度で乗算する
                const targetRot = (mouseX / window.innerWidth) * 360;
                // イージングの公式を用いて滑らかにする
                // 値 += (目標値 - 現在の値) * 減速値
                rot += (targetRot - rot) * 0.02;

                // ラジアンに変換する
                const radian = rot * Math.PI / 180;
                // 角度に応じてカメラの位置を設定
                camera.position.x = 1000 * Math.sin(radian);
                camera.position.z = 1000 * Math.cos(radian);
                // 原点方向を見つめる
                camera.lookAt(new THREE.Vector3(0, 0, 0));


                // レンダリング
                renderer.render(scene, camera);

                requestAnimationFrame(tick);
            }
        }
    </script>
</head>
<body>
    <canvas id="myCanvas"></canvas>
</body>
</html>
\end{lstlisting}

\subsection{05-31.html 大量のオブジェクトを描画してみよう}

\begin{lstlisting}[caption=05-31.html]
<!DOCTYPE html>
<html>
  <head>
    <meta charset="utf-8" />
    <script src="https://unpkg.com/three@0.137.4/build/three.min.js"></script>
    <script src="https://unpkg.com/stats.js@0.17.0/build/stats.min.js"></script>
    <script>
      // ページの読み込みを待つ
      window.addEventListener('DOMContentLoaded', init);

      function init() {
        // サイズを指定
        const width = 960;
        const height = 540;
        // 1辺あたりに配置するオブジェクトの個数
        const CELL_NUM = 20;

        // レンダラーを作成
        const renderer = new THREE.WebGLRenderer({
          canvas: document.querySelector('#myCanvas'),
        });
        renderer.setPixelRatio(window.devicePixelRatio);
        renderer.setSize(width, height);

        // シーンを作成
        const scene = new THREE.Scene();

        // カメラを作成
        const camera = new THREE.PerspectiveCamera(45, width / height);
        camera.position.set(0, 0, 400);

        const container = new THREE.Group();
        scene.add(container);

        // 共通マテリアル
        const material = new THREE.MeshNormalMaterial();
        // Box
        for (let i = 0; i < CELL_NUM; i++) {
          for (let j = 0; j < CELL_NUM; j++) {
            for (let k = 0; k < CELL_NUM; k++) {
              // 立方体個別の要素を作成
              const mesh = new THREE.Mesh(new THREE.BoxGeometry(5, 5, 5), material);

              // XYZ座標を設定
              mesh.position.set(10 * (i - CELL_NUM / 2), 10 * (j - CELL_NUM / 2), 10 * (k - CELL_NUM / 2));

              // メッシュを3D空間に追加
              container.add(mesh);
            }
          }
        }

        // フレームレートの数値を画面に表示
        const stats = new Stats();
        stats.domElement.style.position = 'absolute';
        stats.domElement.style.top = '0px';
        document.body.appendChild(stats.domElement);

        tick();

        // 毎フレーム時に実行されるループイベントです
        function tick() {
          container.rotation.x += Math.PI / 180;
          container.rotation.y += Math.PI / 180;

          // レンダリング
          renderer.render(scene, camera);

          // レンダリング情報を画面に表示
          document.getElementById('info').innerHTML = JSON.stringify(renderer.info.render, '', '    ');

          // フレームレートを表示
          stats.update();

          requestAnimationFrame(tick);
        }
      }
    </script>
  </head>
  <body>
    <canvas id="myCanvas"></canvas>
    <pre id="info"></pre>
  </body>
</html>
\end{lstlisting}

\subsection{05-32.html 最適化してみよう}
\begin{lstlisting}[caption=05-32.html]
<!DOCTYPE html>
<html>
  <head>
    <meta charset="utf-8" />
    <script src="https://unpkg.com/three@0.137.4/build/three.min.js"></script>
    <script src="https://unpkg.com/three@0.137.4/examples/js/utils/BufferGeometryUtils.js"></script>
    <script src="https://unpkg.com/stats.js@0.17.0/build/stats.min.js"></script>

    <script>
      // ページの読み込みを待つ
      window.addEventListener('DOMContentLoaded', init);

      function init() {
        // サイズを指定
        const width = 960;
        const height = 540;

        // レンダラーを作成
        const renderer = new THREE.WebGLRenderer({
          canvas: document.querySelector('#myCanvas'),
        });
        renderer.setPixelRatio(window.devicePixelRatio);
        renderer.setSize(width, height);

        // シーンを作成
        const scene = new THREE.Scene();

        // カメラを作成
        const camera = new THREE.PerspectiveCamera(45, width / height);
        camera.position.set(0, 0, 400);

        // 1辺あたりに配置するオブジェクトの個数
        const CELL_NUM = 25;
        // 結合用のジオメトリを格納する配列
        const boxes = [];
        for (let i = 0; i < CELL_NUM; i++) {
          for (let j = 0; j < CELL_NUM; j++) {
            for (let k = 0; k < CELL_NUM; k++) {
              // 立方体個別の要素を作成
              const geometryBox = new THREE.BoxGeometry(5, 5, 5);

              // 座標調整
              const geometryTranslated = geometryBox.translate(
                10 * (i - CELL_NUM / 2),
                10 * (j - CELL_NUM / 2),
                10 * (k - CELL_NUM / 2)
              );

              // ジオメトリを保存
              boxes.push(geometryTranslated);
            }
          }
        }
        // ジオメトリを生成
        const geometry = THREE.BufferGeometryUtils.mergeBufferGeometries(boxes);

        // マテリアルを作成
        const material = new THREE.MeshNormalMaterial();
        // メッシュを作成
        const mesh = new THREE.Mesh(geometry, material);
        scene.add(mesh);

        // フレームレートの数値を画面に表示
        const stats = new Stats();
        stats.domElement.style.position = 'absolute';
        stats.domElement.style.top = '10px';
        document.body.appendChild(stats.domElement);

        tick();

        // 毎フレーム時に実行されるループイベントです
        function tick() {
          mesh.rotation.x += Math.PI / 180;
          mesh.rotation.y += Math.PI / 180;

          // レンダリング
          renderer.render(scene, camera);

          // レンダリング情報を画面に表示
          document.getElementById('info').innerHTML = JSON.stringify(renderer.info.render, '', '    ');

          // フレームレートを表示
          stats.update();

          requestAnimationFrame(tick);
        }
      }
    </script>
  </head>
  <body>
    <canvas id="myCanvas"></canvas>
    <pre id="info"></pre>
  </body>
</html>
\end{lstlisting}

\section{できちゃった人}
オブジェクト指向を使った

https://ics.media/tutorial-three/class/

https://ics.media/tutorial-three/class\_method/

をトライしてみよう。


\flushright{以上}


\end{document}